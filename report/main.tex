\documentclass{article}

% To use unicode character
\usepackage[utf8]{inputenc}

% For clickable links
\usepackage{hyperref}

% To control margins
\usepackage[margin=1.5in]{geometry}

% For bibliography
\usepackage[square,numbers]{natbib}
\bibliographystyle{abbrvnat}

% For pictures
\usepackage{graphicx}

% To typeset math
\usepackage{amsthm}
\usepackage{amsmath}
\usepackage{amsfonts}
\usepackage{bbm} % For indicator function

% To typeset code
\usepackage{listings}

% Well, it's in the name
\usepackage{algorithm}
\usepackage[noend]{algpseudocode}


\DeclareMathOperator*{\argmax}{arg\,max}

% ====================================
\title{On Upper-Confidence Bound Policies for Non-Stationary Bandit Problems}
\author{Maxence Giraud, Mathis Demay}
\date{2020}

\begin{document}

\maketitle

\begin{abstract}

	In this report we sum up the work of Garivier et al. \cite{garivier2008upperconfidence} and try to extend their work to new possibilities. This paper is about finding an algorithm for the non-stationary bandit problem. The authors study two algorithms, both based on \textit{Upper-Confidence Bound}(UCB) policies. The first, already existing algorithm is \textit{Discounted-UCB}(D-UCB), and they propose another one, \textit{Sliding-Window UCB}(SW-UCB). We implemented both policies and try to compare to two new variants we came up with of \textit{Sliding-Window UCB}(SW-UCB).
	
\end{abstract}

\section{Introduction}

	\subsection{Multi armed bandits and UCB1 algorithm}

		The setting is the following. TODO

A well-known, efficient algorithm to solve this problem is known as \textit{UCB-1}, and is described in Algorithm \ref{UCB1.}

\begin{algorithm}
    \caption{UCB1}
    For t from 1 to $K$, play arm $I_t = t$ \\
    For t from $K$+1 to $T$, play arm 
    $$ I_t = \argmax_{1\leq i \leq K} \bar X_t(i) + c_t(i)$$
\end{algorithm}


	\subsection{Non Stationary Bandits}

		The work discussed in this paper is focused on Non-Stationary bandit, as of now we have only seen Stationary bandits. In this problem the learner must decide which arm to play with the same reward system as in the classic Multi-armed bandit problem but facing the possibility of a changing environment.\\

This problem can be expressed in the following way. The rewards $X_s(i)$ of an arm $i$ are modeled by an independent sequence of random variables from a distribution that may change across time.\\

Garivier et al. focused their work on abruptly changing environment, meaning that the distributions of rewards remain constant during periods and change at unknown time instants called \textbf{breakpoints}. Other non Stationary bandits problems exist where the distributions of the rewards are changed continuously but are not the main topic of this paper. They nevertheless still try to see how the different policies perform in both types of non stationary environments.\\

It is straighforward that standard policies such as UCB reviewed in the last subsection are not appropriate for this kind of environment, this is why the authors compare a new algorithm of their own Sliding Window UCB with an existing one Discounted UCB. Those two policies are going to be the main focus of this report.

\section{Discounted UCB}
	\begin{algorithm}
    \caption{Discounted UCB}
    \label{alg:d_ucb}
    For t from 1 to $K$, play arm $I_t = t$ \\
    For t from $K$+1 to $T$, play arm 
    $$ I_t = \argmax_{1\leq i \leq K} \bar X_t(\gamma,i) + c_t(\gamma,i)$$

\end{algorithm}

\section{Sliding Window UCB}

	\begin{algorithm}
    \caption{Sliding Window UCB}
    \label{alg:sw_ucb}
    For t from 1 to $K$, play arm $I_t = t$ \\
    For t from $K$+1 to $T$, play arm 
    $$ I_t = \argmax_{1\leq i \leq K} \bar X_t(\tau,i) + c_t(\tau,i)$$
\end{algorithm}

Again, $\beta(t,i)=\bar{X}_t(i)+c_t(i)$.
However, about the definition of $\bar{X}$ given in the paper, we reckon there is a problem in the definitions. In the definition of $X$ there is an $N_t(\gamma, i)$ but no $\gamma$ in the sum, however $\gamma$ appears in the sum of $N_t$. Then the definition of $\bar{X_t}$ is not homogeneous in terms of $\gamma$. Also, in the definition of $c_t$, $N_t$ appears with a parameter $\tau$, not $\gamma$. Also, the sum in $N_t$ begins at $t=1$, not considering the same amount of terms as $X_t$. We concluded the definitions of $\bar{X_t}$ and $c_t$ are right but $N_t$ is a wrong copy-paste. Then we propose the following definitions:
\begin{align}
\bar{X}(\tau, i) &= \frac{1}{N_t(\tau, i)}\sum_{s=t-\tau+1}^t X_s(i)\mathbbm{1}(I_s=i) \\
N_t(\gamma, i) &= \sum_{s=t-\tau+1}^t \mathbbm{1}(I_s=i) \\
c_t(\tau, i) &= B\sqrt{\frac{\xi \log(min(t, \tau))}{N_t(\tau,i)}}
\end{align}

These definitions drop $\gamma$, taking into account the fact that they say their variant is 'more abrupt [than \textit{D-UCB}]', and that $\gamma$ only appears in $N_t$. However, it could be interesting to see if the combination of the discount factor and the sliding window is interesting, but they unfortunately don't tackle this question.

[TODO] More properties ?[]

\section{Experiments} \label{sec_expes}

Our implementation can be found here : \url{https://github.com/MaxenceGiraud/ucb-nonstationary} 
	
	\subsection{Our experiments}
	
		In order to deal with the ill-defined SW-UCB, we try $3$ possibilities for this algorithm. Anyway, whether the definition is wrong or not is not a big deal anymore; we find it interesting to explore these $3$ possibilities.

First we choose to work with $\tau > K$. This hypothesis may sound a little arbitrary, however it is coherent in the sense that looking at less time steps than the actual number of arms does not look like a very reliable solution. The exploitation part has to look at several past events in order to be relevant. In that sense we can also say that $\tau$ should be a multiple of $K$, say $5K$ for example. 

Then, with this hypothesis $\tau > K$ we can do something

The first possibility is to 

\section{Critique}
	
	We have several remarks regarding the paper.

The obvious lack of rereading of the definition of \textit{SW-UCD} is definitely a pity, because it is at the root of their main contribution and the subsequent analysis and other paragraphs which follow.

In general, the paper is not very clear. We soon get lost into computational details, without the authors helping us get the big picture out of it. At least, this is what we think, comparing with other papers. For example, if they had described the idea behind SW-UCB more, we might have understood their point.  This difficulty to follow is enhanced by the length of their demonstrations. However they tried to make them easier to follow by breaking them down into different steps.

Still, the analyses of the algorithms are exhaustive.

They could have compared the performances of UCB1 and the others on the stationary environment.

Concerning the experiments, for D-UCB and SW-UCB they took the hyperparameters tuned to the very problem they are working on, and for UCB1 they took theoretical values from another paper. This is criticizable. 

Still, for reproducibility they gave parameters so it is reproducible, more than in many other papers. We couldn't reproduce their results, but this is more likely due to other reasons (such as errors in the code, maybe).

\section{Conclusion}

	The main contribution of the paper is the Sliding Window-UCB algorithm, along with the analysis of the latter and Discounted-UCB.\\

Our main critique is that the paper is not always clear.\\
According to them, D-UCB and SW-UCB perform well on changing environments, unlike UCB1. In practice, we confirmed that UCB1 is the best on constant environments, but in abruptly changing environments it is harder to tell.

\bibliography{nsucb}

\end{document}
