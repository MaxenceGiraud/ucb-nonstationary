After reviewing the Discounted UCB algorithm, the authors propose a new algorithm called Sliding Window UCB. This time instead of relying on some discounted values of past reward, the policy is only insterested in the last $\tau$ rewards. Conceptually this algorithm is very much suited to the non-stationary scenario as it will disregard old drawings of arms. The full algorithm is described in Algorithm \ref{alg:sw_ucb}.\\

\begin{algorithm}
    \caption{Sliding Window UCB}
    \label{alg:sw_ucb}
    For t from 1 to $K$, play arm $I_t = t$ \\
    For t from $K$+1 to $T$, play arm 
    $$ I_t = \argmax_{1\leq i \leq K} \bar X_t(\tau,i) + c_t(\tau,i)$$
\end{algorithm}

Again, $\beta(t,i)=\bar{X}_t(i)+c_t(i)$.
However, about the definition of $\bar{X}$ given in the paper, we reckon there is a problem in the definitions. In the definition of $X$ there is an $N_t(\gamma, i)$ but no $\gamma$ in the sum, however $\gamma$ appears in the sum of $N_t$. Then the definition of $\bar{X_t}$ is not homogeneous in terms of $\gamma$. Also, in the definition of $c_t$, $N_t$ appears with a parameter $\tau$, not $\gamma$. Also, the sum in $N_t$ begins at $t=1$, not considering the same amount of terms as $X_t$. We concluded the definitions of $\bar{X_t}$ and $c_t$ are right but $N_t$ is a wrong copy-paste. Then we propose the following definitions:
\begin{align}
\bar{X}(\tau, i) &= \frac{1}{N_t(\tau, i)}\sum_{s=t-\tau+1}^t X_s(i)\mathbbm{1}(I_s=i) \\
N_t(\gamma, i) &= \sum_{s=t-\tau+1}^t \mathbbm{1}(I_s=i) \\
c_t(\tau, i) &= B\sqrt{\frac{\xi \log(min(t, \tau))}{N_t(\tau,i)}}
\end{align}

These definitions drop $\gamma$, taking into account the fact that they say their variant is 'more abrupt [than \textit{D-UCB}]'. However, it could be interesting to see if the combination of the discount factor and the sliding window is interesting, but they unfortunately don't tackle this question.

We eventually found in \cite{slidesGarivier} that our definition of $N_t$ was good.

Also, they don't deal with the case of an arm which has not been selected in the last $\tau$ steps. In the current formulation, this leads to $N_t(\gamma, i)=0$ and to a division by $0$ in the definition of $\bar{X}(\tau, i)$. We found different ways to cope with this situation, as explained in section \ref{sec_expes}.

[TODO] More properties ?[]